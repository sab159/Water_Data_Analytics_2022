% Options for packages loaded elsewhere
\PassOptionsToPackage{unicode}{hyperref}
\PassOptionsToPackage{hyphens}{url}
%
\documentclass[
]{article}
\title{Assignment 2: Physical Properties of Lakes}
\author{Sophia Bryson}
\date{}

\usepackage{amsmath,amssymb}
\usepackage{lmodern}
\usepackage{iftex}
\ifPDFTeX
  \usepackage[T1]{fontenc}
  \usepackage[utf8]{inputenc}
  \usepackage{textcomp} % provide euro and other symbols
\else % if luatex or xetex
  \usepackage{unicode-math}
  \defaultfontfeatures{Scale=MatchLowercase}
  \defaultfontfeatures[\rmfamily]{Ligatures=TeX,Scale=1}
\fi
% Use upquote if available, for straight quotes in verbatim environments
\IfFileExists{upquote.sty}{\usepackage{upquote}}{}
\IfFileExists{microtype.sty}{% use microtype if available
  \usepackage[]{microtype}
  \UseMicrotypeSet[protrusion]{basicmath} % disable protrusion for tt fonts
}{}
\makeatletter
\@ifundefined{KOMAClassName}{% if non-KOMA class
  \IfFileExists{parskip.sty}{%
    \usepackage{parskip}
  }{% else
    \setlength{\parindent}{0pt}
    \setlength{\parskip}{6pt plus 2pt minus 1pt}}
}{% if KOMA class
  \KOMAoptions{parskip=half}}
\makeatother
\usepackage{xcolor}
\IfFileExists{xurl.sty}{\usepackage{xurl}}{} % add URL line breaks if available
\IfFileExists{bookmark.sty}{\usepackage{bookmark}}{\usepackage{hyperref}}
\hypersetup{
  pdftitle={Assignment 2: Physical Properties of Lakes},
  pdfauthor={Sophia Bryson},
  hidelinks,
  pdfcreator={LaTeX via pandoc}}
\urlstyle{same} % disable monospaced font for URLs
\usepackage[margin=2.54cm]{geometry}
\usepackage{color}
\usepackage{fancyvrb}
\newcommand{\VerbBar}{|}
\newcommand{\VERB}{\Verb[commandchars=\\\{\}]}
\DefineVerbatimEnvironment{Highlighting}{Verbatim}{commandchars=\\\{\}}
% Add ',fontsize=\small' for more characters per line
\usepackage{framed}
\definecolor{shadecolor}{RGB}{248,248,248}
\newenvironment{Shaded}{\begin{snugshade}}{\end{snugshade}}
\newcommand{\AlertTok}[1]{\textcolor[rgb]{0.94,0.16,0.16}{#1}}
\newcommand{\AnnotationTok}[1]{\textcolor[rgb]{0.56,0.35,0.01}{\textbf{\textit{#1}}}}
\newcommand{\AttributeTok}[1]{\textcolor[rgb]{0.77,0.63,0.00}{#1}}
\newcommand{\BaseNTok}[1]{\textcolor[rgb]{0.00,0.00,0.81}{#1}}
\newcommand{\BuiltInTok}[1]{#1}
\newcommand{\CharTok}[1]{\textcolor[rgb]{0.31,0.60,0.02}{#1}}
\newcommand{\CommentTok}[1]{\textcolor[rgb]{0.56,0.35,0.01}{\textit{#1}}}
\newcommand{\CommentVarTok}[1]{\textcolor[rgb]{0.56,0.35,0.01}{\textbf{\textit{#1}}}}
\newcommand{\ConstantTok}[1]{\textcolor[rgb]{0.00,0.00,0.00}{#1}}
\newcommand{\ControlFlowTok}[1]{\textcolor[rgb]{0.13,0.29,0.53}{\textbf{#1}}}
\newcommand{\DataTypeTok}[1]{\textcolor[rgb]{0.13,0.29,0.53}{#1}}
\newcommand{\DecValTok}[1]{\textcolor[rgb]{0.00,0.00,0.81}{#1}}
\newcommand{\DocumentationTok}[1]{\textcolor[rgb]{0.56,0.35,0.01}{\textbf{\textit{#1}}}}
\newcommand{\ErrorTok}[1]{\textcolor[rgb]{0.64,0.00,0.00}{\textbf{#1}}}
\newcommand{\ExtensionTok}[1]{#1}
\newcommand{\FloatTok}[1]{\textcolor[rgb]{0.00,0.00,0.81}{#1}}
\newcommand{\FunctionTok}[1]{\textcolor[rgb]{0.00,0.00,0.00}{#1}}
\newcommand{\ImportTok}[1]{#1}
\newcommand{\InformationTok}[1]{\textcolor[rgb]{0.56,0.35,0.01}{\textbf{\textit{#1}}}}
\newcommand{\KeywordTok}[1]{\textcolor[rgb]{0.13,0.29,0.53}{\textbf{#1}}}
\newcommand{\NormalTok}[1]{#1}
\newcommand{\OperatorTok}[1]{\textcolor[rgb]{0.81,0.36,0.00}{\textbf{#1}}}
\newcommand{\OtherTok}[1]{\textcolor[rgb]{0.56,0.35,0.01}{#1}}
\newcommand{\PreprocessorTok}[1]{\textcolor[rgb]{0.56,0.35,0.01}{\textit{#1}}}
\newcommand{\RegionMarkerTok}[1]{#1}
\newcommand{\SpecialCharTok}[1]{\textcolor[rgb]{0.00,0.00,0.00}{#1}}
\newcommand{\SpecialStringTok}[1]{\textcolor[rgb]{0.31,0.60,0.02}{#1}}
\newcommand{\StringTok}[1]{\textcolor[rgb]{0.31,0.60,0.02}{#1}}
\newcommand{\VariableTok}[1]{\textcolor[rgb]{0.00,0.00,0.00}{#1}}
\newcommand{\VerbatimStringTok}[1]{\textcolor[rgb]{0.31,0.60,0.02}{#1}}
\newcommand{\WarningTok}[1]{\textcolor[rgb]{0.56,0.35,0.01}{\textbf{\textit{#1}}}}
\usepackage{graphicx}
\makeatletter
\def\maxwidth{\ifdim\Gin@nat@width>\linewidth\linewidth\else\Gin@nat@width\fi}
\def\maxheight{\ifdim\Gin@nat@height>\textheight\textheight\else\Gin@nat@height\fi}
\makeatother
% Scale images if necessary, so that they will not overflow the page
% margins by default, and it is still possible to overwrite the defaults
% using explicit options in \includegraphics[width, height, ...]{}
\setkeys{Gin}{width=\maxwidth,height=\maxheight,keepaspectratio}
% Set default figure placement to htbp
\makeatletter
\def\fps@figure{htbp}
\makeatother
\usepackage[normalem]{ulem}
% Avoid problems with \sout in headers with hyperref
\pdfstringdefDisableCommands{\renewcommand{\sout}{}}
\setlength{\emergencystretch}{3em} % prevent overfull lines
\providecommand{\tightlist}{%
  \setlength{\itemsep}{0pt}\setlength{\parskip}{0pt}}
\setcounter{secnumdepth}{-\maxdimen} % remove section numbering
\ifLuaTeX
  \usepackage{selnolig}  % disable illegal ligatures
\fi

\begin{document}
\maketitle

\hypertarget{overview}{%
\subsection{OVERVIEW}\label{overview}}

This exercise accompanies the lessons in Water Data Analytics on the
physical properties of lakes.

\hypertarget{directions}{%
\subsection{Directions}\label{directions}}

\begin{enumerate}
\def\labelenumi{\arabic{enumi}.}
\tightlist
\item
  Change ``Student Name'' on line 3 (above) with your name.
\item
  Work through the steps, \textbf{creating code and output} that fulfill
  each instruction.
\item
  Be sure to \textbf{answer the questions} in this assignment document.
\item
  When you have completed the assignment, \textbf{Knit} the text and
  code into a single PDF file.
\item
  After completing your assignment, fill out the assignment completion
  survey in \sout{Sakai} Google Survey link.
\end{enumerate}

Having trouble? See the assignment's answer key if you need a hint.
Please try to complete the assignment without the key as much as
possible - this is where the learning happens!

Target due date: 2022-02-01 (updated 20 January 2022)

\hypertarget{setup}{%
\subsection{Setup}\label{setup}}

\begin{enumerate}
\def\labelenumi{\arabic{enumi}.}
\tightlist
\item
  Verify your working directory is set to the R project file,
\item
  Load the tidyverse, lubridate, and rLakeAnalyzer packages
\item
  Import the NTL-LTER physical lake dataset and set the date column to
  the date format.
\item
  Using the \texttt{mutate} function, add a column called Month. Remove
  temperature NAs.
\item
  Set your ggplot theme (can be theme\_classic or something else)
\end{enumerate}

\begin{Shaded}
\begin{Highlighting}[]
\CommentTok{\# Check working directory}
\FunctionTok{getwd}\NormalTok{() }\CommentTok{\#should be the R project file}
\end{Highlighting}
\end{Shaded}

\begin{verbatim}
## [1] "C:/Users/lenovo/Desktop/Duke/Spring 2022/ENV790/Water_Data_Analytics_2022"
\end{verbatim}

\begin{Shaded}
\begin{Highlighting}[]
\CommentTok{\# Load packages}
\FunctionTok{library}\NormalTok{(tidyverse)}
\end{Highlighting}
\end{Shaded}

\begin{verbatim}
## -- Attaching packages --------------------------------------- tidyverse 1.3.1 --
\end{verbatim}

\begin{verbatim}
## v ggplot2 3.3.5     v purrr   0.3.4
## v tibble  3.1.3     v dplyr   1.0.7
## v tidyr   1.1.3     v stringr 1.4.0
## v readr   2.0.0     v forcats 0.5.1
\end{verbatim}

\begin{verbatim}
## -- Conflicts ------------------------------------------ tidyverse_conflicts() --
## x dplyr::filter() masks stats::filter()
## x dplyr::lag()    masks stats::lag()
\end{verbatim}

\begin{Shaded}
\begin{Highlighting}[]
\FunctionTok{library}\NormalTok{(lubridate)}
\end{Highlighting}
\end{Shaded}

\begin{verbatim}
## 
## Attaching package: 'lubridate'
\end{verbatim}

\begin{verbatim}
## The following objects are masked from 'package:base':
## 
##     date, intersect, setdiff, union
\end{verbatim}

\begin{Shaded}
\begin{Highlighting}[]
\FunctionTok{library}\NormalTok{(rLakeAnalyzer)}
\end{Highlighting}
\end{Shaded}

\begin{verbatim}
## Warning: package 'rLakeAnalyzer' was built under R version 4.1.2
\end{verbatim}

\begin{Shaded}
\begin{Highlighting}[]
\FunctionTok{library}\NormalTok{(wesanderson) }\CommentTok{\#fun wes anderson inspired color palettes: }
\end{Highlighting}
\end{Shaded}

\begin{verbatim}
## Warning: package 'wesanderson' was built under R version 4.1.2
\end{verbatim}

\begin{Shaded}
\begin{Highlighting}[]
\CommentTok{\# Load dataset {-} NTL{-}LTER Physical }
\NormalTok{NTLdata }\OtherTok{\textless{}{-}} \FunctionTok{read.csv}\NormalTok{(}\StringTok{"./Data/Raw/NTL{-}LTER\_Lake\_ChemistryPhysics\_Raw.csv"}\NormalTok{)}

\CommentTok{\# Data prep}
   \CommentTok{\# Format date column}
\NormalTok{   NTLdata}\SpecialCharTok{$}\NormalTok{sampledate }\OtherTok{\textless{}{-}} \FunctionTok{as.Date}\NormalTok{(NTLdata}\SpecialCharTok{$}\NormalTok{sampledate, }\AttributeTok{format =} \StringTok{"\%m/\%d/\%y"}\NormalTok{)}

   \CommentTok{\# Add month column}
\NormalTok{   NTLdata }\OtherTok{\textless{}{-}}\NormalTok{ NTLdata }\SpecialCharTok{\%\textgreater{}\%} \FunctionTok{mutate}\NormalTok{(}\AttributeTok{month =} \FunctionTok{month}\NormalTok{(sampledate))}
   
   \CommentTok{\# remove temperature NAs}
\NormalTok{   NTLdata }\OtherTok{\textless{}{-}}\NormalTok{ NTLdata }\SpecialCharTok{\%\textgreater{}\%} \FunctionTok{drop\_na}\NormalTok{(temperature\_C)}
   
\CommentTok{\# Set ggplot theme}
\FunctionTok{theme\_set}\NormalTok{(}\FunctionTok{theme\_minimal}\NormalTok{())}
\end{Highlighting}
\end{Shaded}

\hypertarget{creating-and-analyzing-lake-temperature-profiles}{%
\subsection{Creating and analyzing lake temperature
profiles}\label{creating-and-analyzing-lake-temperature-profiles}}

\begin{enumerate}
\def\labelenumi{\arabic{enumi}.}
\setcounter{enumi}{4}
\tightlist
\item
  For the year 1993, plot temperature and dissolved oxygen profiles for
  all six lakes in the dataset (as two separate ggplots). Use the
  \texttt{facet\_wrap} function to plot each lake as a separate panel in
  the plot. Plot day of year as your color aesthetic and use a reverse y
  scale to represent depth.
\end{enumerate}

What seasonal trends do you observe, and do these manifest differently
in each lake?

\begin{Shaded}
\begin{Highlighting}[]
\NormalTok{lakes\_temp }\OtherTok{\textless{}{-}} \FunctionTok{ggplot}\NormalTok{(}\FunctionTok{subset}\NormalTok{(NTLdata, year4 }\SpecialCharTok{==} \DecValTok{1993}\NormalTok{), }\FunctionTok{aes}\NormalTok{(}\AttributeTok{x =}\NormalTok{ temperature\_C, }\AttributeTok{y =}\NormalTok{ depth, }\AttributeTok{color =}\NormalTok{ daynum)) }\SpecialCharTok{+}
              \FunctionTok{scale\_y\_reverse}\NormalTok{() }\SpecialCharTok{+}
              \FunctionTok{geom\_point}\NormalTok{(}\AttributeTok{alpha =} \FloatTok{0.5}\NormalTok{) }\SpecialCharTok{+} 
              \FunctionTok{scale\_color\_viridis\_c}\NormalTok{() }\SpecialCharTok{+} 
              \FunctionTok{labs}\NormalTok{(}\AttributeTok{title =} \StringTok{"Temperature profiles of NTL{-}LTER lakes"}\NormalTok{,}
                   \AttributeTok{x =} \StringTok{"Temperature (C)"}\NormalTok{, }\AttributeTok{y =} \StringTok{"Depth (m)"}\NormalTok{, }\AttributeTok{color =} \StringTok{"DOY"}\NormalTok{) }\SpecialCharTok{+}
              \FunctionTok{facet\_wrap}\NormalTok{(}\SpecialCharTok{\textasciitilde{}}\NormalTok{ lakename)}

\NormalTok{lakes\_temp}
\end{Highlighting}
\end{Shaded}

\includegraphics{A02_LakesPhysical_files/figure-latex/unnamed-chunk-2-1.pdf}

\begin{Shaded}
\begin{Highlighting}[]
\NormalTok{lakes\_DO }\OtherTok{\textless{}{-}} \FunctionTok{ggplot}\NormalTok{(}\FunctionTok{subset}\NormalTok{(NTLdata, year4 }\SpecialCharTok{==} \DecValTok{1993}\NormalTok{), }\FunctionTok{aes}\NormalTok{(}\AttributeTok{x =}\NormalTok{ dissolvedOxygen, }\AttributeTok{y =}\NormalTok{ depth, }\AttributeTok{color =}\NormalTok{ daynum)) }\SpecialCharTok{+}
            \FunctionTok{scale\_y\_reverse}\NormalTok{() }\SpecialCharTok{+}
            \FunctionTok{geom\_point}\NormalTok{(}\AttributeTok{alpha =} \FloatTok{0.5}\NormalTok{) }\SpecialCharTok{+} 
            \FunctionTok{scale\_color\_viridis\_c}\NormalTok{() }\SpecialCharTok{+} 
            \FunctionTok{labs}\NormalTok{(}\AttributeTok{title =} \StringTok{"Dissolved oxygen profiles of NTL{-}LTER lakes"}\NormalTok{,}
                 \AttributeTok{x =} \StringTok{"Dissolved Oxygen (mg/L)"}\NormalTok{, }\AttributeTok{y =} \StringTok{"Depth (m)"}\NormalTok{, }\AttributeTok{color =} \StringTok{"DOY"}\NormalTok{) }\SpecialCharTok{+}
            \FunctionTok{facet\_wrap}\NormalTok{(}\SpecialCharTok{\textasciitilde{}}\NormalTok{ lakename)}

\NormalTok{lakes\_DO}
\end{Highlighting}
\end{Shaded}

\begin{verbatim}
## Warning: Removed 8 rows containing missing values (geom_point).
\end{verbatim}

\includegraphics{A02_LakesPhysical_files/figure-latex/unnamed-chunk-3-1.pdf}
Seasonal trends are evident in the temperature and dissolved oxygen
profiles of the lakes throughout the year in 1993, with differences
observable both within individual lakes across the seasons and between
the various lakes. Temperatures overall are higher in the warmer midyear
months, and stratification is most pronounced during this time.
Temperatures overall are cooler in the late fall, winter, and early
spring, with stratification disappearing entirely in lakes where late
season measurements are reported. Trends in dissolved oxygen are
similar, though DO does not go to zero when stratification gives way to
mixing. The pronouncement of the deep chlorophyll later appears to
decrease throughout the year as stratification weakens.

\begin{enumerate}
\def\labelenumi{\arabic{enumi}.}
\setcounter{enumi}{5}
\tightlist
\item
  Create a new dataset that calculates thermocline depths for all lakes
  on all dates (hint: you will need group by lake, year, month, DOY, and
  sample date).
\end{enumerate}

\begin{Shaded}
\begin{Highlighting}[]
\NormalTok{thermocline\_depths }\OtherTok{\textless{}{-}}\NormalTok{ NTLdata }\SpecialCharTok{\%\textgreater{}\%} \FunctionTok{group\_by}\NormalTok{(lakename, year4, month, daynum, sampledate) }\SpecialCharTok{\%\textgreater{}\%}
                                  \FunctionTok{summarise}\NormalTok{(}\AttributeTok{thermocline =} \FunctionTok{thermo.depth}\NormalTok{(}\AttributeTok{wtr =}\NormalTok{ temperature\_C, }\AttributeTok{depths =}\NormalTok{ depth, }\AttributeTok{seasonal =} \ConstantTok{FALSE}\NormalTok{)) }\SpecialCharTok{\%\textgreater{}\%}
                                  \FunctionTok{filter}\NormalTok{(thermocline }\SpecialCharTok{\textgreater{}} \DecValTok{1}\NormalTok{) }\CommentTok{\#drop thermoclines within 1m of surface {-}likely resulting from transient stratification due to solar warming. }
\end{Highlighting}
\end{Shaded}

\begin{verbatim}
## `summarise()` has grouped output by 'lakename', 'year4', 'month', 'daynum'. You can override using the `.groups` argument.
\end{verbatim}

\begin{enumerate}
\def\labelenumi{\arabic{enumi}.}
\setcounter{enumi}{6}
\tightlist
\item
  Plot thermocline depth by day of year for your newly made dataset.
  Color each point by lake name, make the points 50\% transparent, and
  choose a color palette other than the ggplot default.
\end{enumerate}

\begin{Shaded}
\begin{Highlighting}[]
\NormalTok{thermocline\_plot }\OtherTok{\textless{}{-}} \FunctionTok{ggplot}\NormalTok{(thermocline\_depths, }\FunctionTok{aes}\NormalTok{(}\AttributeTok{x =}\NormalTok{ daynum, }\AttributeTok{y =}\NormalTok{ thermocline, }\AttributeTok{color =}\NormalTok{ lakename)) }\SpecialCharTok{+} 
                    \FunctionTok{geom\_point}\NormalTok{(}\AttributeTok{alpha =} \FloatTok{0.5}\NormalTok{) }\SpecialCharTok{+}
                    \FunctionTok{scale\_color\_manual}\NormalTok{(}\AttributeTok{values =} \FunctionTok{c}\NormalTok{(wes\_palettes}\SpecialCharTok{$}\NormalTok{Darjeeling1, wes\_palettes}\SpecialCharTok{$}\NormalTok{Darjeeling2)) }\SpecialCharTok{+} 
                    \FunctionTok{labs}\NormalTok{(}\AttributeTok{title =} \StringTok{"Thermocline depth of LTER lakes"}\NormalTok{, }\AttributeTok{x =} \StringTok{"DOY"}\NormalTok{, }\AttributeTok{y =} \StringTok{"Thermocline depth (m)"}\NormalTok{) }
                    
\NormalTok{thermocline\_plot}
\end{Highlighting}
\end{Shaded}

\includegraphics{A02_LakesPhysical_files/figure-latex/unnamed-chunk-5-1.pdf}

\begin{enumerate}
\def\labelenumi{\arabic{enumi}.}
\setcounter{enumi}{7}
\tightlist
\item
  Create a boxplot of thermocline depth distributions split up by lake
  name on the x axis and by month as the fill color (hint: you will need
  to set Month as a factor). Choose a color palette other than the
  ggplot default, relabel axes and legend, and place the legend on the
  top of the graph.
\end{enumerate}

\begin{Shaded}
\begin{Highlighting}[]
\NormalTok{thermo\_boxplot }\OtherTok{\textless{}{-}} \FunctionTok{ggplot}\NormalTok{(thermocline\_depths, }\FunctionTok{aes}\NormalTok{(}\AttributeTok{x =}\NormalTok{ lakename, }\AttributeTok{y =}\NormalTok{ thermocline, }\AttributeTok{fill =} \FunctionTok{as.factor}\NormalTok{(month))) }\SpecialCharTok{+}
                  \FunctionTok{geom\_boxplot}\NormalTok{() }\SpecialCharTok{+} 
                  \FunctionTok{scale\_fill\_manual}\NormalTok{(}\AttributeTok{values =} \FunctionTok{c}\NormalTok{(wes\_palettes}\SpecialCharTok{$}\NormalTok{Royal1, wes\_palettes}\SpecialCharTok{$}\NormalTok{Moonrise2)) }\SpecialCharTok{+} 
                  \FunctionTok{labs}\NormalTok{(}\AttributeTok{title =} \StringTok{"Thermocline depth of LTER lakes"}\NormalTok{, }\AttributeTok{x =} \StringTok{"Lake name"}\NormalTok{, }\AttributeTok{y =} \StringTok{"Thermocline depth (m)"}\NormalTok{, }\AttributeTok{fill =} \StringTok{"Numeric month"}\NormalTok{) }\SpecialCharTok{+}
                  \FunctionTok{theme}\NormalTok{(}\AttributeTok{legend.position =} \StringTok{"top"}\NormalTok{, }\AttributeTok{legend.direction =} \StringTok{"horizontal"}\NormalTok{) }\CommentTok{\#how to get this aligned more nicely?? }

\NormalTok{thermo\_boxplot}
\end{Highlighting}
\end{Shaded}

\includegraphics{A02_LakesPhysical_files/figure-latex/unnamed-chunk-6-1.pdf}

Do all the lakes have a similar seasonal progression of thermocline
deepening? Which lakes have the deepest thermoclines, and how does this
relate to their maximum depth?

\begin{quote}
While the relative positions of the thermocline at different points in
the year are similar across lakes (deepest in later months), the actual
depth and rate of change differ between lakes and between years. The
deepest thermoclines are, unsurprisingly, found in the deepest lakes,
including Peter and Crampton lakes.
\end{quote}

\end{document}
